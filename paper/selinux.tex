\ac{selinux} is a Linux feature that provides a mechanism for supporting access control security policies, including mandatory access controls, through the use of \ac{lsm} in the Linux kernel.
It is not a Linux distribution, but rather a set of Kernel modifications and user-space tools that can be added to various Linux distributions.
Its architecture strives to separate enforcement of security decisions from the security policy itself and streamlines the volume of software charged with security policy enforcement 

\ac{selinux} can potentially control which activities are allowed for each user, process and daemon, with very precise specifications.
However, it is mostly used to confine daemons like database engines or web servers that have more clearly defined data access and activity rights.
A confined daemon that becomes compromised is thus limited in the harm it can do.
Ordinary user processes often run in the unconfined domain, not restricted by \ac{selinux} but still restricted by the classic Linux access rights.
\cite{wiki:selinux}
